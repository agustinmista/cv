\documentclass[helvetica,notitle,flagCMYK,totpages,11pt]{europecv}
\usepackage[T1]{fontenc}
\usepackage{graphicx}
\usepackage[a4paper,top=0.5cm,left=2cm,right=2cm,bottom=2cm]{geometry}
\usepackage[english]{babel}
\usepackage{url}
\usepackage{hyperref}

\linespread{1}
\setlength{\parskip}{1em}
\setlength\parindent{0pt}
\selectlanguage{english}
\pagenumbering{gobble}

\title{
  \begin{flushright}
    \Large{Agustín Mista}\\
    \small{\url{amista@dcc.fceia.unr.edu.ar}}\\
    % \small{Universidad Nacional de Rosario}\\
    \small{Rosario, Argentina}\\
    \small{\today}
  \end{flushright}
  \vspace{-50pt}
}

\date{} 


\begin{document}

\maketitle 

Dear Octopi project faculty members,

I am writing to express my interest in the PhD student position at Chalmers
University of Technology you are offering.
%
I am a Licentiate Student in Computer Science (five years program, equivalent to
a M.Sc. degree) at the National University of Rosario, in Rosario, Argentina.
%
I am currently waiting the approval for the presentation of my thesis, and
expecting to graduate in June of the current year.
%
I am confident that my computer science background along with my research
experience makes me an ideal candidate for your open position.


I have a strong background in functional programming, specially in Haskell,
since it is the main programming language used at many courses of my ongoing
licentiate degree.
%
Over the past two years, I spent my time working on penetration testing in
Haskell. Starting with an internship at CIFASIS-CONICET supervised by Gustavo
Grieco, where I became one of the main developers of QuickFuzz, a prolific
experimental fuzzer written in Haskell and based on QuickCheck.
%
There, I worked mainly developing meta-programming tools to generate random data
based on external libraries interfaces.
%
This internship gave me the opportunity of co-authoring an article published in
the Journal of Systems and Software, addressing the state-of-the-art techniques
used in the development of QuickFuzz to exploit datatype-driven generation of
random data.
%
Moreover, in mid-2107 I started working with Professor Alejandro Russo on a
meta-programming tool to derive optimized QuickCheck generators using a
predictive approach.
%
In this light, I had the opportunity to visit Chalmers University in October of
2017, with the purpose of writing an article in conjunction with Alejandro Russo
and John Hughes.
%
This article is currently under revision at the 2018 International Conference on
Functional Programming, and describes the theoretical an empirical results
involved in the development of DRaGen, the resulting tool of the joint work with
Alejandro.
%
My licentiate thesis essentially consists of an extended version of this
article.


I think that software testing is a critical task that can be greatly automated
in the presence of systems developed using a strongly typed language like
Haskell, exploiting the statically known compile-time information.
%
A research target highly relevant to the Octopi project, considering the current
exponential growth of, barely verified, unsafe IoT devices developed using
weakly typed programming languages.


Apart from my training in computer science, I consider myself an electronics
enthusiast, with two years of background in electronics engineering, and
experience in programming embedded devices like the AVR or PIC families of RISC
microcontrollers.
%
A skill I think could benefit the evolution of the Octopi project.


I believe that I am a suitable candidate for the PhD position in view of my
expertise on functional programming, penetration testing and embedded devices
programming.
%
I hope therefore, that on consideration of my CV and the additional information
I am providing, you will be persuaded of my potential to perform well and make a
real contribution to the Octopi project.
%
In particular, my expertise ideally fits the Track E. (\emph{{Penetration
    Testing}}) of the project, which I would like to contribute to.
%
However, I would also be comfortable working on Track B (\emph{Compilation and
  runtime}), given my knowledge on low level programming of constrained devices.


Thank you very much for taking the time to review my application. I am really
looking forward to hearing from you. Feel free to contact me at anytime for
further information.


\vfill
Kind regards,\\
Agustín Mista

\end{document} 
