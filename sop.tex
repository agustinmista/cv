\documentclass[helvetica,openbib,logo,notitle,flagCMYK,totpages]{europecv}
\usepackage[T1]{fontenc}
\usepackage{graphicx}
\usepackage[a4paper,top=1.27cm,left=3.5cm,right=3.5cm,bottom=2cm]{geometry}
\usepackage[english]{babel}
\usepackage{url}
\usepackage{hyperref}

\linespread{1.2}
\setlength{\parskip}{1em}


\title{Statement of Purpose}
\author{Claudio Agust\'in Mista}
\date{}

\begin{document}
\selectlanguage{english}

\maketitle 
   
    Four years of Computer Science study and a successful student internship at
    Argentina's scientific research institute led me to where I am today:
    wanting to become an active member of the scientific community, interested
    in research fields such as Theory of Programming Languages and Systems
    Security. 

    My personal interest in the academia begun right after I was exposed to the
    concept of Functional Programming for the first time at a data structures
    course in my university (more precisely with the Haskell programming
    language), wondering myself about the many advantages that comes with a
    programming language with such strong mathematical underlying machinery.
    Many concepts of the category theory that I had to study before now have a
    \emph{useful} application in real world software! 

    My first approach to the scientific research ecosystem was at my student
    internship CIFAFIS-CONICET under the supervision of Gustavo Grieco and
    Mart\'in Ceresa, where I actively participated improving QuickFuzz, a
    Haskell written experimental grammar fuzzer able to find security
    compromising bugs in real world applications and libraries. In particular,
    I have studied and implemented a technique to automatically derive random
    value generators of third-party libraries using metaprogramming. This
    improvements allowed me to find and report previously unknown security
    concerning bugs on libraries used by big software systems such as Mozilla
    Thunderbird and the GNOME Project. During this internship, we also wrote an
    article submitted to the Journal of Systems and Software, describing the
    state of the art techniques used and the results obtained in the later
    stages of development of QuickFuzz. This article is currently under
    revision. 

    After finishing my student internship, I kept working at CIFASIS-CONICET
    where I am doing relevant research to my Master's thesis, related to the
    structural minimization of bug inducing test cases.   

    Once I finish my Master in Computer Science studies, I am interested in
    applying for a PhD. position where I could make active research on the
    fields of my interest. In particular, I aim to make a progress in how
    functional programming and software security can converge leading in better
    and safer systems. 

    %Since my geographical situation does not offer many academic events of this
    %degree of relevance, 
    I believe that having an opportunity to assist to the Cornell, Maryland,
    Max Planck Pre-doctoral Research School would be an important step in the
    direction I want my academic career to evolve.

\end{document} 
